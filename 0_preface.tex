\chapter{前言}
\label{chap:preface}
现在市面上已经出版的中文 \LaTeX{} 入门书有胡伟老师的《\LaTeXe{} 完全学习手册》,
还有刘海洋前辈的《\LaTeX{} 入门》;网络上流传的中文入门资料,比较优秀的,
则有黄新刚的《\LaTeX{} 笔记》、台湾李果正老师的《大家来学 \LaTeX{}》以及
不才编写的\href{http://liam0205.me/2014/09/08/latex-introduction/}%
  {\uline{一份其实很短的 \LaTeX{} 入门文档}}。
这些入门资料都不错,但任然不断有同好表示 \LaTeX{} 入门难。
这让我感到,仅仅有这些资料是不够的,遂动了再编写一份对新人更友好的入门资料的念头。

这本书的灵感来自于 Zed A.\,Shaw 所著的《Learn Python the Hard Way》。
在这本书中,Shaw 精心设计了一系列的练习题,将 Python 的基本概念穿插其中,
让读者在练习的过程中,逐渐上手 Python。
诚然,这本书没有完全介绍 Python 的所有语言特性,也没有涉及到太多的用法技巧。
但是,Shaw 通过练习,将编程的精要娓娓道来。
我相信,哪怕是对编程毫无所知的人,仔细阅读并认真实践了书中每一个练习后,
都能具备基础的代码编写能力。

如 Shaw 在书中所言,所谓「笨方法」指得是教授的方式,而不是教授的内容。
在阅读本书的时候,你需要
\begin{enumerate}
  \item 做每一道习题;
  \item 一字不差地写出每一份手稿;
  \item 编译手稿,得到正确的结果。
\end{enumerate}
刚开始这对你来说会非常难,但你需要坚持下去。如果你通读了这本书,每晚花个一两小时做做习题,
你可以为自己读下一本 \LaTeX{} 书籍打下良好的基础。
通过这本书,你不可能学到 \LaTeX{} 的方方面面,但是将能习得最基本的学习方法。

这本书的目的是教会你 \LaTeX{} 新手所需的三种最重要的技能:读和写、注重细节、发现不同。

\section*{读和写}
\label{sec:read_and_write}

在 \LaTeX{} 中,有许多形如 \cs{command}\oarg{oarg}\marg{marg}
的\emph{控制序列}(\emph{命令})。
显然,如果你连这些带着特殊符号的命令都打不出来,那就别想学好 \LaTeX{} 了。

为了让你记住各种符号的名字并对它们熟悉起来,你需要将代码写下来并且运行起来。
这个过程也会让你对 \LaTeX{} 更加熟悉。

\section*{注重细节}
\label{sec:the_details}

注重细节的程度,几乎是任何行业评判雇员能力的通用标准。
是否注重细节以及注重细节的程度,决定了你是否能顺利从本书毕业,也决定了你编写的手稿
排版出来的效果。
如果你不够重视细节,那么,也许你能够用 \LaTeX{} 排版出一些手稿,但是在排版的要求上,
你的手稿可能问题百出。

你需要将本书里的示例一字不差地打出来。通过这样的实践,你才能训练自己将精力集中在细节上的能力。

\section*{发现不同}
\label{sec:the_difference}

在你对着示例或者做练习的时候,不可避免地,你会打错一些命令。
我希望你不会因为频繁地见到错误提示而沮丧——要知道,这是不可避免的,哪怕是十分有经验的 \LaTeX{}
使用者,也会出错。
你的任务,是把自己打的东西和示例/标准答案做对比,找出任何细微的不同,然后把所有的差异都修正好。

这样的练习,会让你对手稿里的错误和缺陷更加敏感。

\section*{不要复制粘贴}
\label{sec:no_copy_and_paste}

复制粘贴本书提供的代码,就和学生时代抄袭作业一样,是一种自欺欺人的行为。我希望你不要这样做。

本书的练习都经过了仔细的设计,目的是锻炼你的大脑和双手,让你有能力读代码、理解代码、边写代码。
如果复制粘贴本书提供的代码,那么一切就失去了意义。

\section*{许可协议}
\label{sec:license}

版权所有 侵权必究 (C) 2016 by Liam Huang.

你可以在不收取任何费用,而且不修改任何内容的前提下自由分发这本书给任何人。
但是本书的内容只允许完整原封不动地进行分发和传播。
也就是说如果你用这本书给人上课,只要你不向学生收费,
而且给他们看的书是完整未加修改的,那就没问题。
